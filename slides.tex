\documentclass[10pt]{beamer}

\usepackage{cmap}
\usepackage[T2A]{fontenc}
\usepackage[utf8]{inputenc}
\usepackage{amsfonts}
\usepackage{amsthm}
\usepackage{mathtools}
\usepackage{color}
\usepackage{hyperref}
\usepackage{graphicx}
\usepackage{pdfpages}
\usepackage{forest}
\usepackage{adjustbox}
\usepackage{times}
\usepackage{tikz}
\usetikzlibrary{decorations.pathreplacing}  % Подключение библиотеки для декораций
\usetikzlibrary{snakes,decorations.pathmorphing}


\mode<presentation>{
    \usetheme{Marburg}
    \usecolortheme{sidebartab}
}

% \newtheorem{theorem}{Theorem}[section]
% \newtheorem{lemma}{Lemma}[section]
% \newtheorem{proposition}{Proposition}[section]
% \newtheorem{corollary}{Corollary}[section]
% \newtheorem{definition}{Definition}[section]
% \newtheorem{remark}{Remark}[section]
% \newtheorem{example}{Example}[section]

\newcommand{\E}{\ensuremath{\mathbb{E}}}
\newcommand{\D}{\ensuremath{\mathbb{D}}}
\newcommand{\C}{\ensuremath{\mathbb{C}}}
\newcommand{\R}{\ensuremath{\mathbb{R}}}
\newcommand{\Q}{\ensuremath{\mathbb{Q}}}

\newcommand{\red}[1]{\textcolor{red}{#1}}
\newcommand{\blue}[1]{\textcolor{blue}{#1}}
\newcommand{\green}[1]{\textcolor{green}{#1}}
\newcommand{\orange}[1]{\textcolor{orange}{#1}}
\newcommand{\teal}[1]{\textcolor{teal}{#1}}
\newcommand{\purple}[1]{\textcolor{purple}{#1}}

\renewcommand{\phi}{\varphi}
\renewcommand{\epsilon}{\varepsilon}
\renewcommand{\le}{\leqslant}
\renewcommand{\ge}{\geqslant}

\begin{document}
    \title[NN-based pricing]{Прайсинг деривативов с помощью ML}
    \author{УРМ}
    \date{\today}
    \institute{Sber}

    \begin{frame}
        \titlepage
    \end{frame}

    \section{Что такое дериватив?}
    \begin{frame}{Что такое дериватив?}

        Это соглашение, по которому стороны обмениваются (возможно случайными) cashflow в некоторые моменты времени в будущем.

        \begin{tikzpicture}[scale=0.9]  % Масштабирование всей картинки
            decoration={snake, amplitude=6mm, segment length=4mm}
            % Ось времени
            \fill[black] (0,0) circle (0.1); % Жирная точка в нуле
            \draw[->] (0,0) -- (10,0) node[right] node[below] {Time}; % Линия оси времени

            % Отметки на оси времени
            
            \foreach \x in {0, 2, 4, 6, 8} {
                \draw (\x,0.1) -- (\x,-0.1) node[below] {\x};
            }
            
            % Волнистые стрелки
            \draw[decorate,decoration={snake, amplitude=0.3mm, segment length=4mm, post length=0.5mm},->,thick,blue] (1,1) -- (1,0);
            \draw[decorate,decoration={snake, amplitude=0mm, segment length=4mm, post length=0.5mm},->,thick,blue] (3.5,1) -- (3.5,0);
            \draw[decorate,decoration={snake, amplitude=0.3mm, segment length=4mm, post length=0.5mm},->,thick,blue] (7,1) -- (7,0);
            
            % Подписи для cashflow
            \node[above] at (1,1) {CF$_1$};
            \node[above] at (3.5,1) {CF$_2$};
            \node[above] at (7,1) {CF$_3$};
        \end{tikzpicture}

        Кроме того бывает такое, что одна из сторон может закончить сделку по некоторым условиям в произвольный момент.

        В общем случае
        \[
            price = \E \left( \sum Cashflow_t \mathcal{DF}(t) \right)
        \]
        где $DF$ ~--- дисконтфактор (возможно стохастический).
    \end{frame}

    \section{Подход с линеаризацией}
    \begin{frame}{Подход с линеаризацией}
        Мы можем разделить состояние рынка и сам контракт дериватива:
        \begin{itemize}
            \item Состояние рынка ~--- различные форвардные кривые, волатильности etc (легко векторизуется);
            \item Дериватив ~--- описание сделки, наиболее общо в json формате.
        \end{itemize}

        Можно попробовать сделать 2 модели:
        \begin{align*}
            & \phi: Markets \to \mathcal{M} = \R^n \\
            & \psi: Derivatives \to \mathcal{L} = \R^n
        \end{align*}
        Оптимизируем модели 
        \[
            \int\limits_{Loans}\int\limits_{Markets}\left( \phi(m) \cdot \psi^T(l) - price(m, l) \right)^2 dm dl \to 0
        \]

        \begin{block}{Remark}
            Естественно мы считаем, что доступна некоторая (не очень быстрая) функция price, которая может прайсить любой дериватив на любом рынке алгоритмически.
        \end{block}
    \end{frame}

    \begin{frame}{Подход с линеаризацией}
        \begin{itemize}
            \item Необходимо исследовать, как доработать подход так, чтобы за $\phi$ и $\psi$ стояли интерпретируемые понятия из финансовой математики 
            ~--- репликация сложного дериватива через линейную комбинацию более простых (отсюда и название этого подхода).
            \item Архитектура универсальна: ее можно сочетать с активным обучением и/или с рекуррентными моделями для эмбеддинга дериватива
        \end{itemize}
    \end{frame}

    \section{Активное обучение}
    \begin{frame}{Active learning}
        В идеале было бы получить следующую архитектуру инференса:
        \begin{itemize}
            \item Кроме цены деритватива модель предсказывает и свою уверенность в этой оценке (подобно тому, как устроен uncertainty в Gaussian Process Regressor);
            \item Таким образом, если модель уверена в своей оценке, то мы будем использовать ее цену;
            \item Если же неуверенность большая, то мы будем отправлять дериватив считаться алгоритмически и дообучать модель на этой точке.
        \end{itemize}
    \end{frame}

    \section{Reinforcement learning}
    \begin{frame}{Reinforcement learning}
        В финансах хорошо известны два применения RL, однако они недоисследованы:
        \begin{itemize}
            \item Возможность досрочного погашения приводит к задаче оптимального управления: 
            ее часто решают с помощью техник Монте Карло, но интересно решить более универсальным методом ~--- RL;
            \item Управление транзакционными издержками также формулируется, как задача оптимального управления.
        \end{itemize}

        \begin{block}{Remark}
            Для исследования этих подходов требуется более глубокое понимание финансовой математики.
        \end{block}
    \end{frame}

\end{document}

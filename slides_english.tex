\documentclass[10pt]{beamer}

\usepackage{cmap}
\usepackage[T2A]{fontenc}
\usepackage[utf8]{inputenc}
\usepackage{amsfonts}
\usepackage{amsthm}
\usepackage{mathtools}
\usepackage{color}
\usepackage{hyperref}
\usepackage{graphicx}
\usepackage{pdfpages}
\usepackage{forest}
\usepackage{adjustbox}
\usepackage{times}
\usepackage{tikz}
\usetikzlibrary{decorations.pathreplacing}
\usetikzlibrary{snakes,decorations.pathmorphing}

\mode<presentation>{
    \usetheme{Marburg}
    \usecolortheme{sidebartab}
}

\newcommand{\E}{\ensuremath{\mathbb{E}}}
\newcommand{\D}{\ensuremath{\mathbb{D}}}
\newcommand{\C}{\ensuremath{\mathbb{C}}}
\newcommand{\R}{\ensuremath{\mathbb{R}}}
\newcommand{\Q}{\ensuremath{\mathbb{Q}}}

\newcommand{\red}[1]{\textcolor{red}{#1}}
\newcommand{\blue}[1]{\textcolor{blue}{#1}}
\newcommand{\green}[1]{\textcolor{green}{#1}}
\newcommand{\orange}[1]{\textcolor{orange}{#1}}
\newcommand{\teal}[1]{\textcolor{teal}{#1}}
\newcommand{\purple}[1]{\textcolor{purple}{#1}}

\renewcommand{\phi}{\varphi}
\renewcommand{\epsilon}{\varepsilon}
\renewcommand{\le}{\leqslant}
\renewcommand{\ge}{\geqslant}

\begin{document}
    \title[NN-based pricing]{Derivative Pricing with Machine Learning}
    \author{Risk Block}
    \date{\today}
    \institute{Sber}

    \begin{frame}
        \titlepage
    \end{frame}

    \section{What is a derivative?}
    \begin{frame}{What is a derivative?}

        A derivative is a contract where parties exchange (possibly random) cash flows at specific future time points.

        \begin{tikzpicture}[scale=0.9]  % Scaling the entire diagram
            decoration={snake, amplitude=6mm, segment length=4mm}
            % Time axis
            \fill[black] (0,0) circle (0.1); % Bold point at zero
            \draw[->] (0,0) -- (10,0) node[right] node[below] {Time}; % Time axis line

            % Ticks on the time axis
            \foreach \x in {0, 2, 4, 6, 8} {
                \draw (\x,0.1) -- (\x,-0.1) node[below] {\x};
            }
            
            % Wavy arrows
            \draw[decorate,decoration={snake, amplitude=0.3mm, segment length=4mm, post length=0.5mm},->,thick,blue] (1,1) -- (1,0);
            \draw[decorate,decoration={snake, amplitude=0mm, segment length=4mm, post length=0.5mm},->,thick,blue] (3.5,1) -- (3.5,0);
            \draw[decorate,decoration={snake, amplitude=0.3mm, segment length=4mm, post length=0.5mm},->,thick,blue] (7,1) -- (7,0);
            
            % Cash flow labels
            \node[above] at (1,1) {CF$_1$};
            \node[above] at (3.5,1) {CF$_2$};
            \node[above] at (7,1) {CF$_3$};
        \end{tikzpicture}

        Additionally, one of the parties may have the right to terminate the contract under certain conditions at any time.
    \end{frame}

    \begin{frame}{What is a derivative?}
        In general,
        \[
            price = \E \left( \sum Cashflow_t \mathcal{DF}(t) \right)
        \]
        where $\mathcal{DF}$ is the discount factor (possibly stochastic).
        Thus, pricing can be done using Monte Carlo simulations or binomial models.
        \begin{itemize}
            \item This is challenging for derivatives of general form.
            \item Computational methods are often slow and cannot process an entire bank's portfolio efficiently.
        \end{itemize}
    \end{frame}

    \section{Active learning}
    \begin{frame}{Active learning}
        Ideally, we aim for the following inference architecture:
        \begin{itemize}
            \item The model not only predicts the derivative price but also estimates its confidence in the prediction (similar to uncertainty modeling in Gaussian Process Regression).
            \item If the model is confident, we use its prediction.
            \item If uncertainty is high, we compute the derivative price algorithmically and use the result to further train the model.
        \end{itemize}
        Next, we present a possible model architecture.
    \end{frame}

    \section{Linearization approach}
    \begin{frame}{Linearization approach}
        We can separate market conditions from the derivative contract itself:
        \begin{itemize}
            \item Market conditions: forward curves, volatilities, etc. (easily vectorized).
            \item Derivative contract: trade description, typically in JSON format.
        \end{itemize}

        We can define two models:
        \begin{align*}
            & \phi: Markets \to \mathcal{M} = \R^n \\
            & \psi: Derivatives \to \mathcal{D} = \R^n
        \end{align*}
        We optimize the models to minimize:
        \[
            \int\limits_{Derivatives}\int\limits_{Markets}\left( \phi(m) \cdot \psi^T(d) - price(m, d) \right)^2 dm dd \to 0
        \]

        \begin{block}{Remark}
            We assume access to a (not very fast) function that can compute derivative prices algorithmically in any market conditions.
        \end{block}
    \end{frame}

    \begin{frame}{Linearization approach}
        \begin{itemize}
            \item Further research is needed to ensure that $\phi$ and $\psi$ have meaningful financial interpretations,
            such as replicating complex derivatives using a linear combination of simpler instruments.
            \item This architecture is flexible: it can be combined with active learning or recurrent models for derivative embedding.
        \end{itemize}
    \end{frame}

    \section{Reinforcement learning}
    \begin{frame}{Reinforcement learning}
        In finance, two applications of RL are well-known but underexplored:
        \begin{itemize}
            \item Early termination rights lead to an optimal control problem, typically solved via Monte Carlo techniques. RL could offer a more general solution.
            \item Managing transaction costs can also be framed as an optimal control problem.
        \end{itemize}

        \begin{block}{Remark}
            Studying these approaches requires a deeper understanding of financial mathematics.
        \end{block}
    \end{frame}

    \section{Team composition}
    \begin{frame}{Team composition}
        Expected team size: 3–6 students.
        
        Required skills:
        \begin{itemize}
            \item Experience with Python and PyTorch.
            \item Probability theory.
        \end{itemize}

        Preferred qualifications:
        \begin{itemize}
            \item Basic knowledge of financial mathematics.
            \item Experience with reinforcement learning.
        \end{itemize}

        \begin{block}{After the project}
            Opportunities include internships, full-time positions, scholarships, and support for publishing research papers.
        \end{block}
    \end{frame}

    \section{Contacts}
    \begin{frame}{Contacts}
        \begin{block}{Vanya Vorobyev}
            Senior Data Scientist
            
            \href{mailto:ievorobev@sberbank.ru}{ievorobev@sberbank.ru}, \href{https://t.me/v0r0bi0v}{@v0r0bi0v}
        \end{block}
    \end{frame}

\end{document}